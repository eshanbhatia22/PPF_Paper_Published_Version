\section{Methodology}
\label{Method}

\subsection{Performance Model}
\label{Method-Model}
For testing and comparing PPF, we used the ChampSim~\cite{Champsim} simulator.
ChampSim is an enhanced version of the framework that was used for the 2nd
Data Prefetch Championship (DPC-2)~\cite{DPC_2}. We model 1-core, 4-core, and
8-core out-of-order machines. The details of the configuration parameters are
summarized in Table~\ref{tab:Sim_params}.

\begin{table}[]
    \centering
    \begin{tabular}{|l|p{3.6cm}|c|}
    \hline
    Level & Configuration & Access Latency \\
    \hline
         L1 Cache & Separate I-cache and D-cache, 32 KB, 8-way & 4 cycles\\
         L2 Cache & Private, 256 KB & 8 cycles\\
         LLC & Shared, 2MB / core, 16-way & 20 cycles\\
         DRAM & 4 GB Single Channel for single-core, 8 GB Double Channel for multi-core & N?? cycles\\
    \hline
    \end{tabular}
    \caption{Memory Model Parameters}
    \label{tab:Sim_params}
\end{table}

% djimenez: changing all the "was" to "is." when describing our work, use
% present tense

The block size is fixed at 64 bytes. Prefetching is only triggered at an L2 cache
demand accesses but could be directed to the L2 or last-level cache. No L1
data level prefetching is done. The LRU replacement policy is used on all levels
of cache hierarchies. Branch prediction is done using the perceptron branch
predictor~\cite{Perc_Branch}.  ThE page size is 4KB.  Champsim operates
all the prefetchers strictly in the physical address space.

\subsection{Testing Under Additional Memory Constraints}
\label{Method-AdditionalMem}
The default single-core configuration simulates a 2MB LLC and a single
channel DRAM with 12.8GB/s bandwidth.  We extend the simulations to
include memory constraints introduced in DPC-2.  Specifically we look
at the following two variations:
\begin{itemize}
\item \textit{Low Bandwidth DRAM}: Here the DRAM bandwidth is limited
  to 3.2 GB/s
\item \textit{Small LLC}: In this scenario, LLC size is reduced to 512
  KB
\end{itemize}
All the multi-core simulations are only done in the default
configuration.

\subsection{Workloads}
\label{Method-Workloads}
% djimenez: don't say this is the first time SPEC 2017 has been used this way.
% i don't know if that's true, and in any event the value is small.

%This is the first time that SPEC 2017 benchmark suite~\cite{SPEC2017}
%has been used to characterize and measure the prefetch performance.
We use the that SPEC 2017 benchmark suite~\cite{SPEC2017}.  We use all the 20
workloads available in the SPEC 2017 suite.  Using the SimPoint~\cite{SimPoint}
methodology, we identified 95 different program segments of 1 Billion
instructions each.

\textit{Single-core performance:} For single-core simulations, we use 200
million instructions to warm-up the microarchitectural structures and the next
one billion instructions to do detailed simulations and collect run-time
statistics. We report the IPC speedup over the baseline of no prefetching.
The final number reported is the geometric mean of the speedup achieved on
individual traces.

\textit{Multi-core performance:} For multi-application workloads, we generate
100 random mixes and another 100 mixes from the memory intensive subset of
SPEC 2017.  For 4-core workload, 200 Million instructions are used for warm-up
and additional 1 Billion instruction simulated for collecting statistics.
Each CPU keeps executing its workload till the last CPU completes one billion
instructions after warm-up.  For collecting IPC and other data, only the first
billion instructions are considered as the region of interest.

Here we report the weighted speedup normalized to baseline
\textit{i.e.}, no prefetching.  For each of the workloads running on a
particular core of the 4-core 8 MB LLC system, we compute
IPC\textsubscript{i}.  We then find the IPC\_isolated\textsubscript{i}
of the same workload running in isolated 1-core 8 MB LLC environment.
Then we calculate the total weighted-IPC for a given workload mix as
$\Sigma$ (IPC\textsubscript{i} / IPC\_isolated\textsubscript{i}).  For
each of the 100 workload-mix, the sum obtained is normalized to the
weighted-IPC calculated similarly for baseline case \textit{i.e.}, no
prefetching, to get the weighted-IPC-speedup.  Finally the geometric
mean of these 100 weighted-IPC-speedup is reported as the effective
speedup obtained by the prefetching scheme.

We repeat the same process for 8-core workloads, correspondingly with 16MB
LLC.  The only difference is that 20 million warm-up instructions and 100
million full instructions are executed.  This is done so as to keep the
simulation run-time within reasonable limits as a single 8-core mix takes up
to 3 days to simulate one billion instructions.

\textit{Validation:} We cross-validated our PPF model using SPEC
2006~\cite{SPEC2006} and CloudSuite~\cite{CloudSuite} benchmarks.  For
single-core SPEC 2006, we developed 94 simpoints spread across all the 29
applications. For multi-core, we followed the same methodology as SPEC 2017.
For CloudSuite, we used the traces made available for the 2nd Cache
Replacement Competition (CRC-2)~\cite{CRC_2}.  The traces include 4 4-core
applications with 6 distinct phases per application.

In total, we used 285 traces representing workloads across 53 applications.
Throughout the paper for SPEC 2017, we consider memory intensive subset as the
traces with LLC MPKI > \_\_.  This includes 48 out of the 95 simpoints
developed.  Since SPEC 2006 differs considerably in terms of memory behaviour,
we define memory intensive subset as traces with LLC MPKI > \_\_.  This
includes 47 out of 94 simpoints.

\subsection{Preferchers Simulated}
\label{Method-Prefetchers}
We compared PPF against three of the latest state of the art hardware-only
prefetchers: Best Offset Prefetcher (BOP), DRAM Aware - Access Map Pattern
Matching (DA-AMPM)~\cite{DA_AMPM} and Signature Path Prefecher (SPP).  BOP was
the winner of 2nd Data Prefetching Championship.  DA-AMPM is the enhanced
version of AMPM, modified to account for DRAM row buffer locality.  SPP has
been shown to outperform BOP on SPEC 2006 traces.  For each of these, we
compare their speedups taking no prefetching as the baseline.

\subsection{Developing Features for PPF}
\label{Method-Features}
This section describes the intuition and analysis that went behind developing
the features.  As noted earlier, we developed a set of nine features that allow
the perceptron layer to correlate prefetching decision with the program
context.  To study the correlation across each feature, we study statistically
the perceptron weights and try to interpret their distribution.

\textbf{Global Pearson's Correlation}\newline For this experiment, we create a
dump of the state of perceptron weights at the end of all trace execution.
The weights obtained from running all the SPEC 2017 traces are concatenated.
Since the weights are collected at the end of individual trace execution, the
perceptron weights have attained a relatively stable value by now.  Hence,
this dump represents a `snap' of the trained perceptron weights across all the
SPEC 2017 traces.  The intuition here is that the feature with bulk of the
perceptron weights concentrated around 0 or small magnitude numbers show a
weak correlation with the prefetching outcome.  On the other hand, features
with most of the weights saturated around highest value (+15) show a high
positive correlation and the features with weights close to the lowest value
(-16) show a strong negative correlation.

We plot a histogram for each feature depicting weights distribution
from -16 to +15 and use this histogram to generate the Pearson's
correlation factor for that feature.  In statistics, Pearson's factor
is a numerical measure of the degree of linear correlation between two
variables.  It ranges from -1 to +1.  The magnitude of Pearson's
factor depicts the extent of correlation and the sign depicts whether
it is a positive correlation or a negative correlation.  Values close
to 0 suggest a low correlation or at times, noisy data.  A value of
+1/-1 suggests a perfectly linear positive / negative correlation
respectively.  Figure XX depicts the histogram and the Pearson's
coefficient for two of the features, YY and ZZ.

Figure XX shows all the features used, arranged in the increasing
order of their Pearson's factor.  As can be seen X out of the 9
features provide a moderate to high correlation, with the magnitude of
P-value > 0.4.  The single most important feature, \_\_ helps provide
a correlation to prefetch outcome with a factor of \_\_.

\textbf{Per Trace Correlation} \newline Another important way to look
at the perceptron features is to see how much their contribution
varies across the traces.  Here we give special attention to features
with low P-values in the previous experiment.  Figure XX shows the
variation of P-values three features : \_\_, \_\_ and \_\_; across all
the SPEC 2017 traces.  For simplicity, the traces are arranged in an
increasing order of contribution made by the feature and the trace
names have been omitted.  It can be seen that even features with a low
overall correlation provide useful correlation (magnitude > 0.4) for
XX out of the 95 simpoints developed for SPEC 2017.  

\textbf{Trimming Features Using Correlation} \newline Besides
providing interesting insights into prefetching behaviour, P-value can
also be used for feature selection and prefetcher tuning.

Here, we introduce the concept of cross-correlation across features.
Just as we examined correlation of each feature with the final
outcome, we can also study correlation between the features.  We used
the above methodology to eliminate features that provide a little or
no information that has already been captured in other features.

As a part of feature selection, we initially came up a mix of 23
features- primary and composite.  By studying cross correlation of
each of these features against others in a 23x23 matrix, we identified
pairs of features with correlation factor > 0.9 in magnitude and
carefully eliminated redundant features.  Using this approach, we
managed to reduce the feature count to 9.  Thus, in the final
implementation of PPF, no two features have a high correlation
between them.  This way we can be sure that each feature is bringing
in contribution that cannot be captured using other features.

\textit{<TODO: VALIDATE THIS>} Secondly, studying the relative
importance of each feature enabled us to vary the number of entries
dedicated for each feature.  Features with higher Pearson's
correlation (like \_\_ and \_\_) were given most importance and
allowed full 12-bits of indexing.  Features like \_\_ and \_\_, with a
low overall P-value were allocated fewer indexes in the feature table.

To conclude, in the above discussion we justify the features for
perceptron from a statistical viewpoint.  We also show how this
information can also be used for prefetcher tuning.  All this study
was made possible only because we used on-line perceptron learning for
prefetching and that enabled us to examine the weights in detail.


\subsection{Overhead for PPF}
\label{Method-Overheads}

\begin{table}[]
    \centering
    \begin{tabular}{|c|c|m{4.8cm}|}
    \hline
        \textbf{Field} &
        \textbf{Bits} &
        \textbf{Comment} \\
    \hline
         Valid & 1 & Indicates a valid entry in the table\\
         Tag & 6 & Identifier for the entry in the table\\
         Useful & 1 & To show if the given entry led to a useful demand fetch\\
         Perc Decision & 1 & Prefetched vs Not-prefetched \\
    \hline
        PC & 12 & \multirow{5}{4.8cm}{Meta-data required for perceptron training}\\
        Address & 24 & \\
        Delta & 7 & \\
        Curr Signature & 12 & \\
        Confidence & 7 & \\
    \hline
        \multicolumn{3}{|c|}{\textbf{Total 84 bits}}\\
    \hline
    \end{tabular}
    \caption{Meta-data Stored in Prefetch Table}
    \label{tab:PTable_metadata}
\end{table}


\begin{table}[]
    \centering
    \begin{tabular}{|c|c|c|c|}
    \hline
        \textbf{Structure} &
        \textbf{Entry} &
        \textbf{Components} &
        \textbf{Total} \\
    \hline
        \multirow{5}{2.2cm}{Signature Table} &      & Valid (1 bit)        &             \\
                                             &      & Tag (16 bits)        &             \\
                                             & 256  & Last Offset (6 bits) & 11008 bits  \\  
                                             &      & Signature (12 bits)  &             \\
                                             &      & LRU (6 bits)         &             \\
    \hline
        \multirow{3}{2.2cm}{~~Pattern Table} &      & $C_{sig}$ (4bits)      &               \\
                                             & 512  & $C_{delta}$ (4*4 bits) & 24576 bits    \\
                                             &      & Delta (4*7 bits)       &               \\
    \hline
        \multirow{3}{1.5cm}{Perceptron\newline}     & 4096*8    &            &              \\
        \multirow{2}{0.9cm}{Weights}                 & 2048*x    & 5 bits     & 164480 bits  \\
                                                    & 128*1     &            &              \\
    \hline
        Prefetch Table\footnotemark[1]             & 1024      & 84 bits    & 86016 bits   \\
    \hline
        Reject Table\footnotemark[2]               & 1024      & 83 bits    & 84992 bits   \\
    \hline
        \multirow{4}{1cm}{Global\newline\newline}   & \multirow{4}{0.2cm}{8} & Signature (12 bits)  & \multirow{4}{1.1cm}{264 bits} \\
        \multirow{3}{1.1cm}{History\newline}        &                        & Confidence (8 bits)  &                               \\
        \multirow{2}{1.2cm}{Register}               &                        & Last Offset (6 bits) &                               \\
                                                    &                        & Delta (7 bits)       &                               \\
    \hline
        Accuracy        & 1     & C$_{total}$       & 10 bits   \\
        Counters        & 1     & C$_{useful}$      & 10 bits   \\
    \hline
        \multirow{3}{1.5cm}{Global PC\newline}      &       & $PC_1$ (12 bits)      &           \\
        \multirow{2}{1.5cm}{~Trackers}              & 3     & $PC_2$ (12 bits)      & 36 bits   \\
                                                    &       & $PC_3$ (12 bits)      &           \\
    \hline
        \multicolumn{4}{|c|}{\textbf{Total: 371392 bits = 45.33 KB}}\\
    \hline
    \end{tabular}
    \caption{SPP-Perc Storage Overhead}
    \label{tab:PPF_overhead}
\end{table}

\footnotetext[1]{Components of Prefetch Table can be found in Table \ref{tab:PTable_metadata}}
\footnotetext[2]{the Reject Table does not need to maintain the useful bit as that only applies for prefetches that ultimately made through}

In this section, we analyze the hardware overhead required to
implement PPF.  The Prefetch Table was enhanced to accommodate
storing of meta-data for perceptron training.  Table
\ref{tab:PTable_metadata} depicts the meta-data stored for each entry in
the Prefetch Table.  Table \ref{tab:PPF_overhead} shows the total
storage overhead of PPF implementation.  The hardware budget for
2nd Data Prefetching championship was 32 KB.  Keeping that in mind 
the considerable speedup PPF obtained over the winner, the extra hardware
budget can be accounted for.  The extra hardware also makes the
overall scheme more scalable than SPP.  In the original SPP paper, it
was demonstrated that adding extra hardware brings little advantage in
terms of performance gain.  The newly added perceptron tables can be
scaled to increase / decrease features depending on the permitted
budget.

In terms of computations, the perceptron mechanism only introduces an
extra adder tree.  The hash perceptron mechanism makes sure than there
is no actual vector multiplication happening in the hardware.
Obtaining the perceptron sum requires addition of 9 5-bit numbers.
Using an adder tree of 4 5-bit adders, this can be done in
ceil($log_{2}9$) = 4 steps.  Perceptron update only requires weight
update by +1 or -1.  Thus, all the operations required for perceptron
inferencing or updating the states of the perceptrons can be
easily done in the time constraints of L2 Cache Accesses.

