\begin{abstract}
Hardware prefetching has been introduced in modern processors as a way to hide
cache latencies.  An efficient prefetcher should be able to identify complex
memory access patterns during program execution. This ability enables the
prefetcher to read a block ahead of its demand access, potentially saving a
cache miss. Accurately identifying the right blocks to prefetch is essential
to achieving high performance from the prefetcher.

In this paper, we introduce Perceptron-based Prefetch Filtering to help make
this prefetching decision accurately. The perceptron layer acts as a check 
to filter out the unnecessary prefetches recommended by the underlying
prefetcher.  We have also explored a range of features that can be used to
train the perceptron layer.  Our results show that perceptron-based filtering
improves performance on the memory intensive subset of the SPEC 2017 benchmark
suite by 6.84\% on single-core and by 11.9\% on multi-core traces, as compared
to a state-of-the art prefetcher.  We also demonstrate that the performance
gained from using our efficient filter scales better with increasing numbers
of cores.

\end{abstract}
