\begin{abstract}

  Hardware prefetching is an effective technique for hiding cache miss
  latencies in modern processor designs. Prefetcher performance can be
  characterized by two main metrics that are generally at odds with
  one another: coverage, the fraction of baseline cache misses which
  the prefetcher brings into the cache; and accuracy, the fraction of
  prefetches which are ultimately used. An overly aggressive
  prefetcher may improve coverage at the cost of reduced accuracy.
  Thus, performance may be harmed by this over-aggressiveness because
  many resources are wasted, including cache capacity and bandwidth.
  An ideal prefetcher would have both high coverage and accuracy.

  In this paper, we introduce Perceptron-based Prefetch Filtering
  (PPF) as a way to increase the coverage of the prefetches generated by
  an underlying prefetcher without negatively impacting accuracy. PPF
  enables more aggressive tuning of the underlying prefetcher,
  leading to increased coverage by filtering out the growing numbers
  of inaccurate prefetches such an aggressive tuning implies. We also
  explore a range of features to use to train PPF's perceptron layer
  to identify inaccurate prefetches. PPF improves performance on a
  memory-intensive subset of the SPEC CPU 2017 benchmarks by 3.78\%
  for a single-core configuration, and by 11.4\% for a 4-core
  configuration, compared to the underlying prefetcher alone.

\end{abstract}
