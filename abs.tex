\begin{abstract}
Hardware prefetching has been introduced in modern
processors as a way to hide cache latencies.  An efficient
prefetcher should be able to identify complex memory access
patterns during a program execution.  This enables it to fetch a
block ahead of its demand access, potentially saving a cache miss.


In this paper, we introduce perceptron learning to help make
this prefetching decision efficiently. The perceptron layer acts 
as a throttler to filter out the unnecessary prefetches recommended 
by the underlying prefetcher.  We have also explored a range 
of features that can be used to train the perceptron layer.  
Our results show that NATCH
improves performance on the memory intensive subset of the SPEC
2017 Benchmark suite by 6.84\% on single-core and by 11.9\% on
multi-core traces, as compared to state-of-the art prefetcher.  
We also demonstrate that the performance gained from using 
our efficient throttler scales better with increasing number 
of cores. 
\end{abstract}
