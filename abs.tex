\begin{abstract}

% djimenez: I have taken out parenthetical terms in favor of
% explicitly mentioning them in the text. some people have a
% visceral reaction against parentheses that delimit things
% essential to understanding the text.

  Hardware prefetching is an effective technique for hiding cache miss
  latencies in modern processor designs. Prefetcher performance can be
  characterized by two main metrics that are generally at odds with
  one another: coverage, the fraction of baseline cache misses which
  the prefetcher brings into the cache; and accuracy, the fraction of
  prefetches which are ultimately used. An overly aggressive
  prefetcher may improve coverage at the cost of reduced accuracy.
  Thus, performance may be harmed by this over-aggressiveness because
  many resources are wasted, including cache capacity and bandwidth.
  An ideal prefetcher would have both high coverage and accuracy.

% Hardware prefetching has been introduced in modern processors as a
% way to hide cache latencies. An efficient prefetcher should
% identify complex memory access patterns during program
% execution. This ability enables the prefetcher to read a block
% ahead of its demand access, potentially preventing a cache
% miss. Accurately identifying the right blocks to prefetch is
% essential to achieving high performance from the prefetcher.

  In this paper, we introduce Perceptron-based Prefetch Filtering
  (PPF) as a way to increase the coverage of the prefetches generated by
  {\color{red}an underlying} prefetcher without negatively impacting accuracy. PPF
  enables more aggressive tuning of {\color{red}the underlying prefetcher},
  leading to increased coverage by filtering out the growing numbers
  of inaccurate prefetches such an aggressive tuning implies. We also
  explore a range of features to use to train PPF's perceptron layer
  to identify inaccurate prefetches. PPF improves performance on a
  memory-intensive subset of the SPEC CPU 2017 benchmarks by 3.78\%
  for a single-core configuration, and by 11.4\% for a 4-core
  configuration, compared to the \color{red}underlying prefetcher alone.


% In this paper, we introduce Perceptron-based Prefetch Filtering to
% help make prefetching decisions accurately. The perceptron layer
% acts as a check to filter out the unnecessary prefetches
% recommended by the underlying prefetcher. We have also explored a
% range of features that can be used to train the perceptron layer.
% Our results show that perceptron-based filtering improves
% performance on the memory intensive subset of the SPEC CPU 2017
% benchmark suite by 6.84\% on single-core and by 11.9\% on
% multi-core traces, as compared to a state-of-the art prefetcher.
% We also demonstrate that the performance gained from using our
% efficient filter continues to scale as the number of cores sharing
% a last level cache increases.

\end{abstract}
