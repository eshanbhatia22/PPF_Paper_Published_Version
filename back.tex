\section{Background Work and Motivation}
\label{Background}

In this section, we compare the existing work done in the domain of
prefetching.  We also discuss about some of the interesting memory
access patterns seen in the workloads in SPEC 2017 benchmark.  The
second half of this section deals with a background of Signature Path
Prefetcher architecture.

\subsection{Related Work}
\label{Background-Related}
The idea of prefetching has been around since Instruction Stream
Buffers were proposed by Jouppi~\cite{ISB}. Some of the earliest data
prefetchers developed to try to identify memory access patterns with a
constant stride pattern~\cite{Smith,Baer,Stride}. Major research was
dedicated to stride distance and depth prediction~\cite{Decoupled,Adaptive}. Newer prefetchers aimed at correlating
predictions with the past memory access addresses~\cite{Address_Correlated,AMPM}.

With time newer and more intricate prefetchers were
introduced~\cite{Wenisch_Temporal_Streaming,Stealth,Feedback_Directed,Coordinated,Bandwidth_Efficient,Pacman,TLB,Linearizing,Sandbox,VLDP,DoL,Domino}.
A whole class of streaming prefetchers developed on the lines of
Spatial / Temporal Memory
Streaming~\cite{Spatial_Pattern,SMS,Temporal_Instruction_Fetch,Off_Chip,STMS,SMS_JILP}.
There have been ideas for control flow speculation directed hardware prefetching~\cite{BFetch,MTBFetch}.

Most modern prefetchers aim at identifying complex memory access
patterns in a given application.  This allows them to capture some of
the irregular patterns seen in pointer chasing data-structures.
Offset based prefetchers are a generalization of the next-line
prefetcher.  Some of the prominent prefetchers exploiting this idea
are Sandbox Prefetcher~\cite{Sandbox} and the Best Offset Prefetcher~\cite{BOP}. 
Another class of prefetchers is lookahead Prefetchers.
Such prefetchers try to speculate deep into the application's memory
access trail.  These include Path Confidence based Lookahead
Prefetching~\cite{SPP} and Kill the Program Counter~\cite{KPC}.

A related line of work also explores cache replacement - insertion
policies and dead block prediction, with an eye towards
prefetching~\cite{DB_Pred,Cache_Burst,KPC,Harmony}.  Nori
\textit{et.al.} introduced prefetching with respect to runtime
criticality~\cite{CATCH}.

\textit{Machine Learning and Prefetching:} Peled \textit{et.al.}
introduce interesting ideas for on-line Reinforcement Learning and
dynamically scaling the magnitude of feedback given to the baseline
prefetcher~\cite{Semantics}. The biggest challenge here is that it
relies on compiler support for getting features to build the context.

The paper by Liao \textit{et. al.} focuses on prefetching for data
center applications~\cite{Datacenter}. They use offline Machine
Learning Algorithms like SVMs and Logistic Regressions to do
parametric search for an optimal prefetcher configuration.

%% NOTE: The below paper is literally the same idea as ours. But poor implementation and no good results. How much to talk about this paper?
Finally, the paper on Data Cache Prefetching with Perceptron
Learning~\cite{BadPerc} talks about the idea of two step prefetching.
The first step is an existing baseline prefetcher while the second
step is a perceptron based throttler.  The paper has lots of
limitations in terms of design and implementation.  That reflects in
the fact that it did not lead to any significant performance gain over
even the basic prefetchers like the Stride Prefetcher~\cite{Stride} and
the Markov Prefetcher~\cite{Markov}.


\subsection{Perceptron Learning in Architecture}
\label{Background-Perceptron}
Perceptron learning for computer architecture design has been around
for a while.  It was first popularized by Jimenez \textit{et. al.} for
doing branch prediction~\cite{Perc_Branch}. The biggest advantage of 
the perceptron model is that it can adjust its weight dynamically 
to increase / decrease the importance given to each feature - branch 
hostories in this case.
The structure that we use is derived from the model introduced in
Piecewise Linear Branch Prediction~\cite{Piece_Linear}.  Here the
feature itself is used to index into a hash of perceptron weights.
This form of indexing makes sure that the hyper-plane learned by the
perceptron weights is able to differentiate between linearly
inseparable outcomes.  \textit{<Is it true? Not sure>}


Perceptron Learning for Reuse Prediction~\cite{Perc_Reuse} extended the
hash perceptron architecture to perform prediction in context of cache
replacement.  Perceptron layer can learn to correlate the cache
replacement behaviour with a variety of features. The features
introduced in that paper were derived from Program Counters of the
current and the last few instructions and from the current memory 
access address.  This led to a
prediction mechanism that is highly accurate, has quick learning rate
and is adaptable to changing program behaviour.  


The work was further extended by Multiperspective Reuse
Prediction~\cite{Multiperspective}, which introduced a varied set of
parametric features.  Although design space exploration for feature
and parameter selection was a non-trivial task, once fixed, it could
give a cache replacement predictor highly suitable for the given set
of applications.

\textit{Deviations From Actual Perceptrons:} Traditionally, a
perceptron prediction involves multiplying the vector of input
features: F\textsubscript{1xN}, with the corresponding weight vector:
W\textsubscript{Nx1} in a dot product fashion to obtain the sum
y\textsubscript{out}.  Here what we use is a perceptron-like
structure.  The feature is used to hash into the weight of perceptrons
and the retrieved weights are added straight away.  This can be seen a
dot product of $\vec{1}$\textsubscript{1xN} with W\textsubscript{Nx1}.
This way, the perceptron algorithm doesn't introduce any
multiplication operations in the inference or training process.  Hence
what we adapt in this work, is a perceptron-like learning algorithm as
it involves the same principle involved in perceptron inferencing and
training.  For simplicity, we will call our implementation as
perceptron implementation.

\textit{Perceptron Update Rule:} In all the above implementations, a
uniform perceptron update principle is being followed.  The weights
need to be updated if the prediction was wrong or the predicted sum
does not cross a certain threshold.  After a certain training period,
these weights are proportional to the probability of the outcome being
true.  This training threshold makes sure that the perceptrons are
trained till a certain level of confidence and yet they are not
over-trained to the given set.  Care is taken that the weights
saturate at certain positive and negative values so that they remain
confined to the bit-width.  The same rules will be applied in the
discussion done in Section \ref{Design-Perceptron} on NATCH learning algorithm.

 
\subsection{Prefetching with SPEC 2017}
\label{Background-SPEC2017}
\textit{<Discuss SPEC 2017 traces here>
<What are the possible graphs / analysis tools we can use for SPEC traces?>}

\subsection{Baseline Prefetcher: The Signature Path Prefetcher}
\label{Background-SPP}

While the basic idea of perceptron based prefetch filtering is applicable 
to any look-ahead prefetcher, we developed a practical implementation 
of the idea using SPP~\cite{SPP} as the underlying prefetcher. 
SPP is a simple, lightweight, lookahead prefetcher which can be easily 
modified to be highly aggressive.  
Here we describe the basic architecture of SPP.

\textbf{Signature Table:} The ST keeps track of 256 most recently accessed
pages.  It is meant to capture memory access patterns within a page
boundary.  SPP indexes into an entry of ST using the page number.  For
each entry corresponding to a page, ST stores the `last block offset'
and `old signature'.  Last block offset is the block offset of the
last memory access of that given page.  The block offset is calculated
with respect to the page boundary.  The signature is a 12-bit
compressed representation of the past few memory accesses for that
page.  The signature is calculated as:
$$New Signature = (\,Old Signature << 3 bits\,) \;\;XOR\;\; (\,Delta\,)$$ 
Delta is the numerical difference between the block offset of the 
current and the previous memory acces. In case a matching page entry 
is found, the stored signature retrieved and used to index into the 
Pattern Table.

\textbf{Pattern Table:} The PT is indexed by the signature generated
from ST.  PT holds predicted delta patterns and their confidence
estimates.  Each entry indexed by the signature holds up to 4 unique
delta predictions.  This is implemented by making PT as a 4-way
associative table.

%% NOTE: Can we do away with this example? 
%% If it looks like we are writing too much about SPP
%% EDIT: Decided to reorganize and remove this for now.
%% 	Consider the case that incoming page number 10 with a block offset 3
%% 	finds a match in the ST.  The retrieved pattern signature is 0x30 and
%% 	the Last Offset is 1.  Since now there is the Last Block Offset (1),
%% 	Incoming Block Offset (3), Old Signature (0x30), and New Signature
%% 	calculated as per above equation (0x182), SPP can infer
%% 	non-speculatively that the given pattern of memory accesses (as
%% 	captured in the Old Signature) leads to the particular Delta.  In
%% 	general, Delta is defined as the difference between the prefetch
%% 	suggestion block and the initial block which triggered the prefetch.
%% 	In this case, since SPP is in learning phase, it is defined as the
%% 	difference between the Incoming Block Offset and the Last Block Offset
%% 	(+2 in this case).  This, in turn generates the new memory pattern
%% 	(New Signature).  This newly learned signature pattern and the delta
%% 	is stored in the Pattern Table.

\textbf{Prefetch Filter:} The PF is a 1024 entry 1-way associative table 
which keeps a record of
last few entries prefetched.  This proves useful for updating the prefetcher 
states if a tracked prefetch leads to a demand hit or a cache eviction. 

\textbf{Lookahead Prefetching:} On each trigger, SPP tries to go down
the program speculation path using its own prefetch suggestion.
Using current prefetch as the baseline, it re-accesses the PT to generate further
prefetches.  It keeps on repeating this cycle of accessing PT and
updating the signature based on highest confidence prefetch from the
last iteration.  These iteration count till which SPP manages to
predict prefetch entries in the lookahead manner is characterized as
its `depth'.  While doing this, SPP also keeps compounding the
confidence in each depth.  Thus as depth increases, overall confidence
keeps decreasing.  

\textbf{Confidence Tracking}: The PT keeps a track of hits to each
signature through a counter C\textsubscript{sig}.  The number of hits
for a given delta per signature are tracked using a counter
C\textsubscript{delta}.  The confidence for a given delta is
approximated through C\textsubscript{d} = C\textsubscript{delta} /
C\textsubscript{sig}.  When SPP enters into a lookahead mode, the path
confidence P\textsubscript{d} is given as:
$$P\textsubscript{d} \;=\; \alpha  \;.\;  C\textsubscript{d}  \;.\;  P\textsubscript{d-1}$$ 
Here $\alpha$ represents the global accuracy, calculated as the ratio of 
the number of prefetches which led to a demand hit to the number of 
prefetches recommended in total. The range of $\alpha$ is [0,1].

`D' is the lookahead depth.  For d = 1, when SPP is in
non-speculative mode, P\textsubscript{0} can be thought of as 1. 
The final P\textsubscript{d} is thresholded against prefetching-threshold
(T\textsubscript{p}) to decide whether to prefetch or not.  If yes,
then P\textsubscript{d} is thresholded against a numerically bigger
fill-threshold (T\textsubscript{f}) to decide whether the prefetch
should be sent to L2 Cache (high confidence prefetch) or Last Level
Cache (low confidence prefetch)

<Insert a comprehensive figure about SPP structures>

<Insert another detailed picture about SPP data-path flow>

\subsection{Case for an On-line Throttler}
\label{Background-Case}
As compared to some of the other state of the art prefetchers [BOP],
SPP is a less aggressive prefetcher.  In the single core
environment, this gives BOP an edge as there is no resource contention
among the different cores.  Hence more aggressive prefetching is bound
to prove beneficial.  As we increase the core count, we observe that
SPP starts outperforming rest of the prefetchers.  This can be
attributed to the fact that each prefetch suggested by SPP is a
carefully calculated one and that prevents cache pollution.
\textit{<Figure showing the variation of SPP vs BOP wrt core count>}
For 4-core applications, SPP suggests XX\% fewer prefetches than BOP
and yet leads to higher IPC.

The above analysis shows that with a more careful throttling
mechanism, any prefetcher like SPP can be tuned to become much more aggressive, leading to
increased coverage.  The onus of maintaining the accuracy now falls on
the independent throttler.  To test the hypothesis, we tuned down SPP
to the minimum possible threshold.  Only by doing this we managed to
increase the prefetch suggestions made by SPP, by XX\%.
\textit{<Comparison of SPP-Unleased with SPP / BOP>} Obviously this
came at a cost of increased DRAM traffic and cache pollution.

Moreover, the on-line confidence mechanism introduced in SPP was very
rudimentary.  It was based on taking a ratio C\textsubscript{d} =
C\textsubscript{delta} / C\textsubscript{sig} as explained previously. This 
confidence was used to make the decision whether to prefetch or
not to prefetch; and which level to prefetch.  While this
approximation was shown to work in the original implementation, we believe that a
better form of generalised on-line decision making was possible.  
Hence, it was
necessary to build a robust and adaptable learning mechanism to accept
/ reject the prefetch suggestions; and to decide the fill level (L2
Cache vs Last Level Cache).

To that effect, we introduce an independent on-line perceptron based throttling mechanism.
