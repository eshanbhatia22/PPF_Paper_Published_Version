\section{Related Work}
\label{related}

\subsection{Perceptrons in Cache Management}

%Perceptron-based prediction mechanisms have proven to yield superior accuracy, 
%which represents an opportunity for optimizations in cache management. 
%Khan \textit{et. al.} proposed SDBP~\cite{sdbp}, in this work, a sampler 
%structure keeps partial tags of sampled sets separate from the cache. 
%Three tables of two-bit saturating counters are accessed using a technique 
%similar to a skewed branch predictor~\cite{Piece_Linear}. For each hit to 
%a block in the sampled set, the program counter (PC) of the relevant memory 
%instruction is hashed into the three tables and the corresponding counters 
%are decremented. For each eviction from a sampled set, the counters corresponding 
%to the PC of the last instruction to access the victim block are incremented. 
%For an LLC access, the predictor is consulted by hashing the PC of the memory 
%access instruction into the three tables and taking the sum of the indexed 
%counters. When the sum exceeds some threshold, the accessed block is 
%predicted to be dead. Tags in sampled sets are managed with true LRU and 
%a reduced associativity, but the LLC may be managed by any policy. 
%The paper applies its predictor to do block replacement and bypass for LLC.

In addition to branch prediction~\cite{PerceptronPredictor},
perceptron-based learning has been applied to the area of cache
management.  Teran \textit{et al.} propose using perceptrons to
predict cache line resuse, bypass, and replacement~\cite{Perc_Reuse}.
Perceptron Learning trains weights selected by hashes of multiple
features, including the PC of the memory access instruction, some
other recent PCs, and two different shifts of the tag of the
referenced block. These features are used to index into weight tables,
and the weights are then thresholded to generate a prediction. When a
block from one of a few sampled sets~\cite{sdbp} is reused or evicted,
the corresponding weights are decremented or incremented, according to
the perceptron learning rule. Multiperspective Reuse
Prediction~\cite{Multiperspective} improves on Perceptron Learning by
contributing many new features.

%features. Compared to previous work that focuses on one or two features at a
%time, Multiperspective prediction uses many features, each contributing to the
%overall prediction. The prediction enables placement, promotion and bypass
%optimizations that further improves performance.

\subsection{Spatial Prefetchers}

% Two widely used prefetchers are Next-$n$ Lines~\cite{nextn} and
% Stride~\cite{stride} prefetchers. Both capture regular memory access
% patterns with low overhead. The Next-$n$ Lines prefetcher queues
% prefetches for the next $n$ blocks after any given miss, expecting
% that the principle of spatial locality will hold and those cache
% blocks after a missed block are likely to be used in the future. The
% stride prefetcher identifies strided reference patterns in programs
% based on past behavior of missing loads.  When a load misses, cache
% blocks ahead of that miss are fetched in the pattern following the
% previous behavior in the hope of avoiding future misses.  Without
% further knowledge about temporal locality and application
% characteristics, these prefetchers cannot do more than detecting and
% prefetching regular memory access patterns with limited spatial
% locality.

Spatial prefetchers include such well-understood examples as the
next-line (or next-$n$-line) prefetcher, and the stream prefetcher,
and are distinguished by prefetching data without regard for the order
in which the data will be accessed.  In addition to these simpler
examples, Somogyi \textit{et al.}  propose Spatial Memory Streaming
(SMS)~\cite{SMS}.  SMS works by learning the spatial footprint of all
data used by a program within a region of memory around a given
missing load, and when the load that causes an new miss elsewhere, the
same spatial footprint is prefetched.  Ishii \textit{et al.} propose
the Access Map Pattern Matching prefetcher (AMPM)~\cite{AMPM}, which
creates a map of all accessed lines within a region of memory, and
then analyzes this map on every access to see if any fixed-stride
pattern can be identified and prefetched that is centered on the
current access. DRAM-Aware AMPM (DA-AMPM)~\cite{DA_AMPM} is a variant
of AMPM that delays some prefetches so they can be issued together
with others in the same DRAM row, increasing bandwidth utilization.
Pugsley \textit{et al.}  propose the Sandbox
Prefetcher~\cite{Sandbox}, which analyzes candidate fixed-offset
prefetchers in a sandboxed environment to determine which is most
suitable for the current program phase.  Michaud proposes the
Best-Offset Prefetcher~\cite{BOP}, which determines the optimal offset
to prefetch while considering memory latency and prefetch timeliness.

%Somogyi et al. proposed a practical, low-overhead prefetcher: Spatial Memory
%Streaming (SMS) prefetcher~\cite{SMS}. SMS leverages code-based correlation to
%take advantage of spatial locality in the applications over larger regions of
%memory called spatial regions. It predicts future access pattern within a
%spatial region around a miss based on a history of access patterns initiated
%by that missing instruction in the past. SMS is effective, but it indirectly
%infers future program control-flow when it speculates on the misses in a
%spatial region. Thus, the state overhead of this predictor can be higher than
%the others in this class. The idea was further extended by the class of
%related prefetchers~\cite{STMS,Temporal_Instruction_Fetch}.

%Other work also considers prefetch timeliness. Michaud proposed the
%Best-Offset prefetcher~\cite{BOP}. A prefetch offset is set automatically and
%dynamically tries to adapt to the application behavior. A good prefetch
%offset is determined by looking at two recent access to a particular block
%that did not happen too recently. The time between both accesses should be
%greater than the latency for completing a prefetch request.

\subsection{Lookahead Prefetchers}

Unlike spatial prefetchers, lookahead prefetchers take program order
into account when they make predictions.  Shevgoor \textit{et al.}
propose the Variable Length Delta Prefetcher (VLDP)~\cite{VLDP}, which
correlates histories of deltas between cache line accesses within
memory pages with the next delta within that page. SPP~\cite{SPP} and
KPC's prefetching component~\cite{KPC} are more recent examples of
lookahead prefetchers. They try to predict not only what data will be
used in the future, but also the precise order in which the data will
be used, within a given page. Predictions made by lookahead
prefetchers can be fed back into their prediction mechanisms to
predict even further down a speculative path of memory acesses. These
prefetchers can also generalize their learned patterns from one page,
and use those patterns to make predictions in other pages.

%Previous prefetchers have only limited ability to capture memory delta
%patterns. Moreover, a good prefetch proves to be useful only if the prefetcher
%manages to bring the block before its next demand access. Both these issues
%are overcome by Lookahead Prefetchers that use their own predictions to create
%the speculative path of the program's memory access pattern. By going down the
%speculation trail to a certain depth, the prefetcher can bring in cache blocks
%well before they are demand accessed by the processor. VLDP~\cite{VLDP}
%prefetches a static depth ahead of the demand fetch, without considering the
%prefetching confidence. More recent Lookahead Prefetchers include SPP and
%KPC~\cite{KPC}

%Talk about DA-AMPM

\subsection{Managing Prefetched Data}

%not sure this is where this section was going, I just kidna went with it. 

A low-accuracy aggressive prefetcher can significantly harm
performance.  To minimize interference from prefetching, Wu \textit{et
  al.} propose PACMan~\cite{pacman}, a prefetch-aware cache management
policy. PACMan dedicates some LLC sets to each of three competing
policies that treat demand and prefetch requests differently, using
the policy in the rest of the cache that shows the fewest
misses. Seshadri \textit{et al.} propose ICP~\cite{icp}, which demotes
a prefetch to the lowest reuse priority on a demand hit, based on the
observation that most prefetches are dead after their first hit. To
address prefetcher-caused cache pollution, it also uses a variation of
EAF~\cite{eaf} to track prefetching accuracy, and inserts only
accurate prefetches to the higher priority position in the LRU
stack. Jain \textit{et al.} propose Harmony~\cite{Harmony} to
accomodate prefetches in their MIN algorithm-inspired Hawkeye cache
management system. Ebrahimi \textit{et al.} introduce
HPAC~\cite{HPAC} which provides a coordinated control between 
multiple prefetchers present in a CMP by looking at the 
prefetcher-induced inter-core interferance.

\subsection{Machine Learning for Prefetching}

Peled \textit{et al.} introduce interesting ideas for on-line
Reinforcement Learning and dynamically scaling the magnitude of
feedback given to the baseline prefetcher~\cite{Semantics}. The
prefetcher relies on compiler support to receive features and build
the context.  Liao \textit{et al.}  focus on prefetching for data
center applications~\cite{Datacenter}.  They use offline machine
learning algorithms such as SVMs and logistic regression to do a
parametric search for an optimal prefetcher configuration. 
Hasheni \textit{et al.}~\cite{LSTM} categorize 
prefetching as a regression problem and use LSTM based Deep 
Learning approach.
