\section{PPF Design and Architecture}
\label{Arch}
% Really need a high-level overview of the technique before you start
% talking about tables!

As previously discussed, ultimately it might be profitable to allow a
given prefetcher to speculate as deeply as possible, since often some
useful prefetches are still being generated long after the confidence
of the prefetcher has fallen below the point at which performance
improvements are seen, due to the increasing rate of inaccurate
prefetches.  In order to allow this deep speculation in the
prefetcher, however, the increasing numbers of bad prefetches must be
filtered out.  To this end, we propose to leverage perceptron-based
machine learning as a mechanism to differentiate between potentially
useful deeply speculated prefetches and likely not-useful ones.  This
Perceptron Prefetch Filter (PPF) is placed between the prefetcher and
the prefetch insertion queue, to ensure that predicted bad prefetches
are not inserted into the memory system, \pg{As shown in figure
  whatever...}.

As discussed in Section~\ref{sec:Background-Perceptron}, perceptrons
are useful, light-weight mechanisms to pull together disparate forms
of information and synthesize a decision from them.  In this case, we
will leverage a variety of different relative characteristics of a
given prefetch, such as speculation depth, page address and offset,
into a binary useful/non-useful prediction.  In the remainder of this
section, we discus the the general structure of PPF for enhancing a
given underlying prefetcher.  We will show that PPF is a generalized
prefetch filtering mechanism which may adapted to any prefetcher with
minimal modifications. %Gino: Are these modifications minimal?

\subsection{The Perceptron Filter}
\label{Arch-Perceptron}


The Perceptron Filter is organized as tables of perceptron weights,
with a separate table of weights for each feature.  A different number
of bits are used for each feature to index into its corresponding
table.
%Gino: repharse below?
Each feature's value is used to index into the corresponding table.
This indexing is done using a different number of bits for each
feature.  At most, 12 bits of a feature are used to index into the
weights table. Certain features require more resolution \textit{i.e.},
the full 12 bits of indexing.  On the other hand, some features
requiring lower resolution require as few as 7 bits of indexing.  The
variable indexing was determined by studying the features and
fine-tuned empirically to achieve a good accuracy vs hardware overhead
trade-off.  Exact details are mentioned under ``Area Overheads'' in
Section~\ref{Method-Overheads}

A single entry in the table corresponds to a perceptron weight.  Each
weight is a 5 bit counter - saturating at -16 and +15 and initialized
to zero at the beginning of a program's execution. Our proposed design
uses nine features, represented in nine perceptron weight tables.

\textbf{Inference}\newline In the ChampSim simulator, the prefetching
method is invoked on every L2 Cache demand access.  At that point, the
base prefetcher has the option to perform a prefetch -- if it does,
then it has a choice of how many cache lines to prefetch.  The
suggested prefetches can be either placed in the L2 cache or the
last-level cache based on the prefetcher's internal confidence
mechanism.

When the base prefetcher is triggered, it begins to suggest candidates
for prefetching. All suggested prefetches are passed through the
perceptron filter to decide if they are qualified for final
prefetching.  The perceptron decides whether to prefetch a candidate
by looking at the microarchitectural state \textit{i.e.} the features,
at that instant. Each feature is hashed to form an index into a table
of up to 4096 entries dedicated for that feature.

Once all the weights are retrieved, they are summed and compared to a
preset threshold ,PERC\_ THRESHOLD\_LO.  Only the prefetch candidates
with perceptron sum higher than the threshold qualify for prefetching.
The prefetches that qualify through the perceptron stage are recorded
in the ``Prefetch Table". The prefetch table is a 1024-entry, direct
mapped structure taht conatains all metadata required to recreate the
features.  %Gino: Recreate the features?
When the feedback of the current prefetched line is available at a
later stage, the stored data is used to train the perceptron.

The sum of the perceptron weights are also used to decide whether to
place teh data in the L2 or the last level cache. All prefetch
candidates that are qualified until this stage are compared against
PERC\_THRESHOLD\_HI to decide the fill level. The two thresholds:
PERC\_THRESHOLD\_LO and PERC \_THRESHOLD\_HI are empirically set.

The perceptron sum can be considered the sum of the individual
contribution of a feature.  The value of each contribution corresponds
to a feature's confidence in the final decision on whether to prefetch
a cache line. By summing the individual contributions, the final
perceptron sum denotes the overall confidence for that prefetch
suggestion.  By comparing the perceptron sum to two different
thresholds, we divide the confidence scale of the prefetch suggestion
into three bins. The first bin corresponds to the lowest confidence
and leads to prefetch candidate being rejected.  The next bin
corresponds to prefetches with a moderate confidence level.  This bin
prefetches directeky to fill the bigger last level cache and prevents
polluting the smaller L2 Cache.  The highest confidence prefetches are
filled in the L2 cache.

In addition to the prefetch table mentioned above, PPF also maintains
a ``Reject Table.''  The reject table is a 1024-entry deep
direct-mapped table.  If a prefetch suggestion is rejected by the
perceptron layer, it is logged into the reject table.  The filer is
used to train the perceptron to %Gino: Filer?
avoid false negatives \textit{i.e.}, cases where prediction was to
reject the prefetch but the prefetch would have been useful.

% djimenez: if it were previously stated then it's redundant here? do we need
% this statement?

%As previously stated all the metadata describing the program state at the
%instant of prefetching is stored in the Prefetch Table or the Reject Table.
%This information is useful when the perceptron needs to be updated
%subsequently.

\textbf{Training}\newline In the prefetching environment, feedback for
a prefetch is received whenever there is an eviction or a demand
access from the L2 Cache.  This action triggers training of the base
prefetcher as well as the perceptron layer.  Training the perceptrons
involves accessing the metadata that was stored in Prefetch Table or
the Reject Table.  The cache lines address of the block being trained
is used to access a table, using 10-bits of the address for indexing
and another 6-bits to perform tag matching.  Once the state of the
program at the time of prefetching is available, it is
%Gino: By "state of the program" are you referring to the state of the features?
used to index into the perceptron weights table.

If the demanded block that triggers the training was tagged as a valid
prefetch in the prefetch table, then the earlier prefetch prediction
was correct.  In that case the perceptron weights are incremented by 1
if the predicted sum does not cross a pre-defined threshold. These
training thresholds are introduced to avoid overfitting of the
perceptron weights to the given program behaviour. These thresholds
are referred to as POS\_UPDATE\_THRESHOLD ($\theta_p$) and
NEG\_UPDATE\_THRESHOLD ($\theta_n$), respectively for the positive and
negative values of training saturation.

If a cache block eviction led to training and the corresponding valid
entry is found in the prefetch table without the block being accessed,
then the prediction made by the perceptron was wrong.  The perceptron
should have ideally rejected the prefetch suggestion as a
low-confidence prefetch.  Here the weights are decremented by 1 to
reflect the misprediction. In either case, weights are saturated at
-16 or +15. %Gino: Describe how these limits were decided?

A secondary training mechanism also kicks in during demand fetches.
Before the demand access triggers the next set of prefetches, the
reject table is checked for a valid entry.  A hit means that the
corresponding cache line was initially suggested by the underlying
prefetch engine, but rejected by the perceptron filter.  Thus, the
perceptron should have been more confident about that particular
prefetch.  Once such a scenario is identified, the
state %Gino: state of execution == value of features?
of the execution at the time of prefetch is retrieved from the reject
table.  The retrieved data is used to index into the various weights
tables of the perceptron and the corresponding values are updated by
+1, saturating between and -16 and +15.  This update reflects
increased confidence for the prediction corresponding to that
prefetch.

This mechanism allows us to exploit a previously lost opportunity.  In
prior perceptron based implementations, and in general, prefetching
algorithms, there is usually no way of knowing the result of not
prefetching a particular line.  Our two-step PPF architecture allows
us to overcome that issue.

\subsection{Generalizing PPF for any Prefetcher}
\label{Arch-Generalizing}
The above discussion of PPF shows that it is highly modular and can be
adapted to be used over any base prefecher for increased prefetch
accuracy.  In general, only two hooks are required between PPF and the
baseline prefetcher. The first is to make sure that all the prefetch
candidates of the prefetcher pass through the perceptron filter and if
qualified, the metadata for perceptron indexing be stored. The second
is needed when the feedback of a prior prefetch is available in form
of a subsequent demand hit or cache eviction. In that case, the stored
metadata needs to be retrieved to update the state of the perceptrons.

In general, PPF can be adapted to a new base prefetcher with only a few modifications.
\begin{itemize}

\item \textbf{Enhancing the Base Prefetcher:} By tuning down any
  internal thresholds to increase its inherent aggressiveness.

\item \textbf{Inferencing and Storing:} All prefetch recommendations
  are tested using the perceptron inferencing algorithm. The
  perceptron's output, \textit{true} or \textit{false}, should be
  saved appropriately, along with all metadata required for perceptron
  indexing.

\item \textbf{Retrieving and Training:} When feedback for a prefetch
  is available, the previously stored metadata can be used to re-index
  into the perceptron entries and increment or decrement the weights.

\item \textbf{Feature Selection:} In Section~\ref{Impl-Features}, we
  show the features used for PPF. Six out of the nine features we
  developed use information derived directly from program execution,
  irrespective of the baseline prefetcher. Beyond that, the feature
  set can be expanded to convey any useful information from the
  prefetcher to the perceptron filter.  The methodology explained in
  Section~\ref{Method-Features} talks about developing a minimal
  feature set for PPF.
%For example, in our implementation ``Confidence,'' ``Current Signature XOR Delta,''
%and ``Page Address XOR Confidence'' are such features.

\end{itemize}
