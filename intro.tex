\section{Introduction}
\label{Introduction}
Processor scaling and memory scaling have been enabled with different
philosophies in mind.  While processor scaling focused on speed improvements,
memory scaling was centred around increasing the storage capacity.  Process
technology reaching scaling limitations led to the Memory Wall~\cite{MemWall}
-- the increasing performance gap between processor speed and the memory
speeds.  Various techniques have been developed to combat this wide gap. Data
prefetching is one such important technique. Data prefetching exploits the
fact that in most applications, memory accesses have a well-defined pattern.

An ideal prefetching scheme woudl capture these memory access patterns to
predict future accesses in a timely manner.  Other important decisions faced
by the prefetcher are when to bring in the prefetch block of data, which level
of the cache hierarchy to place the block in, and what block to evict to
accommodate the incoming block.  Memory accesses patterns can be as simple as
fetching the next cache line or a complex pattern such as pointer chasing.
Predicting future accesses requires a trade-off between coverage and accuracy.
Most prefetchers maintain an internal confidence mechanism by keeping counters
to track events such as cache hits / misses, references to an entry in the
table, etc. By thresholding this confidence against different values, the
degree of aggressiveness can be adjusted.

For a prefetcher to capture highly irregular access patterns, it needs to be
highly aggressive (high coverage).  Thus, the prefetcher would recommend many
useless prefetches, leading to cache pollution (low accuracy).  This effect is
even more pronounced in a multi-core scenario where the last-level cache is a
shared resource~\cite{Friendly}.  Conversely, a conservative prefetcher would
be highly accurate but might not make enough useful prefetches.

We propose Perceptron-based Prefetch Filtering (PPF).  It is an enhancement to
the existing state of the art prefetcher and robustly overcomes the coverage
vs.  accuracy trade-off.  The idea is to use a state-of-the-art prefetcher as
the base engine.  For that we choose Signature Path Prefetcher (SPP) and
modify the design to make it as aggressive as possible.  This modification
enables the base engine to capture complex memory access patterns and go
deeper into the speculation path, leading to increased coverage. Normally such
aggressive prefetching would come at the cost of increased DRAM traffic.
However, prefetch suggestions recommended by the base prefetch engine are
passed through the perceptron based throttler.  Over time, perceptron layer
learns to correlate the prefetch recommendations with various available
features, leading to elimination of useless prefetches. This filtering leads
to overall increased accuracy of the prefetch scheme and reduction in DRAM
traffic.

\vspace{1ex}This paper describes PPF, giving its merits, analysis and
future scope for expansion.  The contributions include:

\begin{itemize}

\item An on-line neural model is used for hardware data
  prefetching.  Previous work in this area either relied on program
  semantics~\cite{Semantics} or were application specific~\cite{Datacenter}.
% djimenez: removing reference to this arXiv paper. we are not obligated to cite it.
%, or
%  could not yield a significant improvement over the baseline prefetcher
%  used~\cite{BadPerc}.

\item Implementing the perceptron throttler in a robust and accurate
  practical prefetching mechanism, yielding a significant performance
  improvement as compared to the other state of the art prefetchers.
  Moreover, the perceptron throttler learns to adapt itself to shared resource
  constraints, leading to further increased performance in multi-core and
  lower capacity environments.

\item Defining a methodology for determining an appropriate set of 
  features for prediction regardless of the underlying prefetcher used.
  More details are explained in Section \ref{Method-Features}.

\end{itemize}

On a single core configuration, PPF increased performance by 6.84\% as
compared to the next best prefetcher, SPP.  On a multi-core system running a
mix of memory intensive subset of the SPEC 2017 traces, PPF saw an improvement
of 11.9\% over SPP for a 4-core system and 10.67\% for an 8-core system.

The paper is organized as follows. Section \ref{Background} describes
background work and summarizes SPP.  Section \ref{Design} discusses the
detailed implementation of the perceptron mechanism and the features used.
Section \ref{Method} gives the evaluation methodology and explores the feature
space for perceptron learning.  Section \ref{Results} evaluates the
performance of the prefetcher and Section \ref{Conclusion} concludes the
paper.
